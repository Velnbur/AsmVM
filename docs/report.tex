\documentclass[a4paper,12pt]{report}

\usepackage[T2A]{fontenc}
\usepackage[utf8]{inputenc}
\usepackage[english,ukrainian]{babel}
\usepackage{amsmath}
\usepackage{ragged2e}
\usepackage{graphicx}
\usepackage{listings}
\usepackage{hyperref}
\graphicspath{ {./} }

\usepackage{titlesec}
\titleformat{\section}[block]{\Large\bfseries\filcenter}{}{1em}{}
\titleformat{\subsection}[block]{\bfseries\filcenter}{}{1em}{}

\author{Кирило Байбула Аленович}
\title{Визначення швидкодії обчислювальної системи}

\begin{document}

\begin{titlepage}
	\begin{center}
		\Large
		\textbf{Київський національний університет імені Т.Шевченка}
		\vspace{5cm}

		\Huge
		\textbf{Звіт}

		\LARGE
		До лабораторної роботи №2
		\vspace{0.5cm}

		\textbf{ІМІТАЦІНА МОДЕЛЬ ПРОЦЕСОРА}
		\vfill
	\end{center}

	\begin{FlushRight}
		Кирило Байбула Аленович \\
		Групa К-21 \\
		Факультету комп’ютерних наук \\
		та кібернетики
	\end{FlushRight}

	\vspace{0.5cm}

	\begin{center}
		\textbf{Київ} \\
		2021
	\end{center}

\end{titlepage}
\clearpage

\section{МЕТА}
Необхідно розробити програмну модель процесора та реалізувати його
імітаційну (тобто комп’ютерну) модель. Мені було запропоновано індивідуальний варіант,
в якому визначена конкретна:
\begin{enumerate}
    \item{Адресність процсеора - 2-адресний;}
    \item{Бітність процесора - 14-бітний;}
    \item{Обов'язкова для реалізації команда процесора - цілочисельне віднімання;}
\end{enumerate}
Також мають обов'язково бути реалізовані:
\begin{enumerate}
    \item{розміщення інтерпретуємої програми у текстовому файлі (наприклад, один рядок=одна команда);}
    \item{мінімум 2 команди (одна з них - занесення значення у регістр, інші задаються варіантом);}
    \item{для операндів/регістрів представлення побітно, можливо, для деяких варіантів із побайтним групуванням бітів;}
    \item{фіксація у регістрі стану  як мінімум знаку ре­зуль­та­ту виконання команди;}
    \item{потактове виконання команд (наприклад, 1-й такт – занесення команди у регістр команди, 2-й такт -  виконання операції і занесення результату).}
\end{enumerate}
\clearpage
\section{НАПИСАННЯ ЛАБОРАТОРНОЇ}
Лабораторна була написана мовою прогамування \textbf{GO} у системі сімейства Unix
. За основу для описання команд у текстовому файлі був взятий синтаксис асемблера
\textbf{nasm}, де \textbf{mov} - оператор занесення значення у регістр та між регістрами
і \textbf{sub} - оператор цілочисельного віднімання. Результат останього заноситься у перший операнд.
Поточний стан процесора у програмі має виглядати як поточна команда,
яку потрібно виконати(IR), список регістрів і побітовому вигляді(R1-R8),
знак останьої резулбтату виконання останьої дії(PS), лічильник виконаних
команд(PC), лічильник тактів(TC);

Програма має у стандартний поток виводу заносити команду яку він збирається
виконувати, поточний стан процесора (його регістри, всі його флаги та лічильники)
та стан після виконання команди. Наприклад, так буде виглядати обробка рядку
\textbf{mov R3, -2}:
\begin{verbatim}
    IR: mov R3, -2
    R1: 00000000000000
    R2: 00000000000000
    R3: 00000000000000
    R4: 00000000000000
    R5: 00000000000000
    R6: 00000000000000
    R7: 00000000000000
    R8: 00000000000000
    PS: +
    PC: 0
    TC: 1

    IR: mov R3, -2
    R1: 00000000000000
    R2: 00000000000000
    R3: 11111111111110
    R4: 00000000000000
    R5: 00000000000000
    R6: 00000000000000
    R7: 00000000000000
    R8: 00000000000000
    PS: -
    PC: 1
    TC: 2
\end{verbatim}
Також, як можна побачити вище, від'ємні числа зберігаться по іншому
на відміну від більшості сучасних Сі-подібних мов програмування.
Для лабораторної числа потрібно було побітово інвертувати і додавати 1,
але, на щастя, \textbf{GO} вже так числа і хранить, тому ніяких додаткових
дій не потрібно було робити.

Приведемо приклад виконання лабораторної для \textbf{sub}:
\begin{verbatim}
    IR: sub R3, 3
    R1: 00000000000000
    R2: 00000000000000
    R3: 11111111111110
    R4: 00000000000000
    R5: 00000000000000
    R6: 00000000000000
    R7: 00000000000000
    R8: 00000000000000
    PS: -
    PC: 1
    TC: 1

    IR: sub R3, 3
    R1: 00000000000000
    R2: 00000000000000
    R3: 11111111111011
    R4: 00000000000000
    R5: 00000000000000
    R6: 00000000000000
    R7: 00000000000000
    R8: 00000000000000
    PS: -
    PC: 2
    TC: 2   
\end{verbatim}
\clearpage
\section{ВИСНОВОК}
Для обробки команд у 2-адресному процессорі достатньо лише двох
тактів. Бітність процессора не впливає на алгоритми виконання команд,
а впливає лише на вмістимість числових регістрів.
У регістрах зберігається вся інформація,
яку обробляє процесор, вони необхідні для виконання команд,
зв’язку процесора з іншими частинами обчислювальної системи,
та більша частина коду програм складається саме з обробки значень
регістрів. Така система роботи не є ідеальною, але через застійлі
стандарти та сучасний стан індустрії такий підхід є необхідним.
Також для реалізації підпрограми нашому процесору потрібний стек та команди
для переходу між мітками програмами та завантаження та вигружання значень
зі стеку.
\subsection{Посилання}
\begin{itemize}
    \item{Код лабораторної: \url{https://github.com/Velnbur/AsmVM.git}}
    \item{Вимоги лабораторної: \url{https://sites.google.com/site/byvkyiv1/arhiteom_stac/arhiteom_lab_06}}
\end{itemize}


\end{document}
